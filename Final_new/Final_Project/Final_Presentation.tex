% starting the actual presentation for Wednesday 
\documentclass{beamer}
\usetheme{Warsaw}

% title page details 
\title{Transfer Learning in Credit Default Prediction}
\subtitle{An Exploratory Study} 
\author{Anthony Bernardi\inst{1}} 
\institute{\inst{1} University of Kentucky - Department of Statistics} 
\date{\today}

\begin{document}

% title page as the first slide 
\begin{frame}
   \titlepage 
\end{frame}

% table of contents 
\begin{frame}{Overview} 
  \tableofcontents
\end{frame} 

% establishing the structure 
\section{Introduction}
\section{Literature Review}
\section{Data and Methodology}
\section{Results and Future Directions}
\section*{References}

% intro slide  
\begin{frame}{Introduction}
  
  \begin{itemize}
    \item What is Transfer Learning?
      \begin{itemize}
        \item DL tool utilizing pre-trained models on a new prediction task 
      \end{itemize}
    \item Why is it relevant and important?
      \begin{itemize}
        \item Reduce computational costs and time 
        \item Helps with limited data sets 
      \end{itemize}
    \item What are its advantages?
      \begin{itemize}
        \item Reduces the need for large data sets 
        \item Can be used for a variety of prediction tasks 
      \end{itemize} 
  \end{itemize}

\end{frame}

% now can get into lit review 
\begin{frame}{Literature Review}

  \begin{itemize}
    \item Papers of Note for this Study 
      \begin{itemize}
        \item \textit{A Concise Review of Transfer Learning} Farahani et al, 2021 
        \item \textit{A Comprehensive Survey on Transfer Learning} Zhuang et al, 2020   
      \end{itemize}
    \item Domain Literature: Credit Default Prediction 
      \begin{itemize}
        \item \text{Kaggle: American Express Credit Default Prediction}
      \end{itemize}
  \end{itemize}

\end{frame} 

% moving into the methodology 
\begin{frame}{Data and Methodology}

  \begin{itemize}
    \item Primary Data Set:
      \begin{itemize}
        \item \href{https://www.kaggle.com/competitions/amex-default-prediction/overview}{American Express Credit Default Prediction}
        \item 1,000,000 + observations! 
      \end{itemize}
    \item Methodology: 
      \begin{itemize}
        \item Data Cleaning and Feature Engineering 
        \item Feather and Parquet Files: Data Compression 
      \end{itemize}
  \end{itemize}

\end{frame}

% initial results and future directions 
\begin{frame}{Initial Results and Future Directions}

  \begin{itemize}
    \item Feature Engineering and Data Cleaning: 
      \begin{itemize}
        \item 200,000 observations after cleaning 
        \item 225 features 
      \end{itemize} 
    \item Initial Model:
      \begin{itemize}
        \item Feed Forward Neural Network 
      \end{itemize}
    \item Baseline Model: 
      \begin{itemize}
        \item Logistic Regression
        \item Random Forest 
      \end{itemize} 
  \end{itemize}

\end{frame}

\begin{frame}[fragile]{Proposed Model Architecture}

  \begin{verbatim}
    class NeuralNetwork(nn.Module):
      def __init__(self):
          super().__init__()
          self.flatten = nn.Flatten() 
          self.linear_relu_stack = nn.Sequential( 
              nn.Linear(225, 24), 
              nn.ReLU(), 
              nn.Linear(24, 225), 
              nn.ReLU(), 
              nn.Linear(225, 1),
              nn.Sigmoid()
          )
  \end{verbatim}

\end{frame}

\begin{frame}[fragile]{Proposed Model Architecture - Feed Forward}

  \begin{verbatim}
    def forward(self, x):
        x = self.flatten(x) 
        logits = self.linear_relu_stack(x) 
        return logits
  \end{verbatim} 

\end{frame} 

\begin{frame}{Future Directions}

    \begin{itemize}
      \item Next Steps: 
        \begin{itemize}
          \item Continue to refine the base NN model 
          \item Compare with baseline models 
          \item Dropout to prevent overfitting
          \item Secondary Task Credit Default Prediction 
        \end{itemize}
      \item Future Ideas: 
        \begin{itemize}
          \item Predicting other types of Default with this pre-trained model 
          \item Comparing with other pre-trained models (e.g. Available models on HuggingFace)
        \end{itemize}
    \end{itemize}

\end{frame} 

% references with links 



\end{document}


